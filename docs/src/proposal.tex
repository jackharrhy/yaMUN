\documentclass[12pt]{article}
\usepackage[utf8]{inputenc}
\usepackage{newtxtext,newtxmath}
\usepackage{indentfirst}

\setlength{\parindent}{10ex}

\begin{document}
\begin{titlepage}
    \begin{center}
        \vspace*{1cm}
            
        \Huge
        \textbf{Group 13 - Project Proposal}
            
        \vspace{0.5cm}
        \LARGE
        A course planning system for MUN students
            
        \vspace{1.5cm}
            
        \textbf{Jack Harrhy}, \texttt{201732922} \texttt{(jaharrhy@mun.ca)} \\
        \textbf{Genadi Valyeyev}, \texttt{201921376} \texttt{(gvalyeyev@mun.ca)}\\
        \textbf{Aidan Langer}, \texttt{201735677} \texttt{(aplanger@mun.ca)}\\
            
        \vfill
    \end{center}
\end{titlepage}
\section{Introduction}
While putting together a course schedule for a semester, it can be tedious to figure out what's being offered for an entire department/semester. You have to view each subject in its own context and then view 
each course with its offerings on its own dedicated page. Even then, once you've found a course, you're told when it occurs on what days, but not if it conflicts with another course you want to take, 
only once you are inserting multiple CRNs into the final step do you become informed there are issues with your selection. \par
Not only is this difficult for people who are used to the system, but for people who are new to MUN, it is a very daunting process.  Our goal is to design a system that makes searching for courses to construct a schedule by different fields more ergonomic and comfortable. To achieve our goal, we need to collect publicly available data from the 
Memorial University website; this includes course descriptions from the University Calendar, information about courses from the course offerings pages and Banner; which are published by the registrar's office each semester; and MUN's People directory.
\section{Proposal}
As mentioned in the introduction, our team wants to design and create a dashboard that would make planning and constructing course schedules more ergonomic and comfortable for Memorial University students. 
Unfortunately, the data we need is not available for download, but published on the university website. In order to work around that, we will need to create a program that would crawl the required pages and extract 
the data using regular expressions. \par
Our system will mainly work with the data that we will collect from the course offering pages. This data contains available campuses, subjects, course names, course codes, sections, CRNs, course dates, rooms, times and days of slots, the amount of credit hours a course provides, and the name of the instructor(s). \par
Additionally, we would like to enrich the course offerings data with course descriptions from the Memorial University calendar and instructor contact information from the "People Search" API that is publicly 
available. We believe that this extra data will provide our users with great convenience, eliminating the need to be constantly switching between web pages to gather information on a course.
\section{Functionalities}
The most essential functionalities of our system revolve around the construction and editing of a user's personalized course list. The primary operation is the ability to list all course offerings and their 
information, allow the user to add whichever ones they choose to their list, and to give them the ability to remove said courses from the list afterwards. \par
Although this is not part of our essential functionality, we intend to allow the user to receive a bounty of useful information, including pulling course descriptions from the university calendar, listing the courses' prerequisites, and collecting instructor's contact 
information from the aforementioned "People Search" API. To browse through the courses, you start from a raw list of all courses, where the user can then choose to sort by a specific attribute (Subject, Professor, Time, etc.), or the user can place a keyword into a search bar and be shown relevant results. This can be searching for a course by name, 
searching for all courses taught by a specific instructor, etc. \par 
Additionally, users will have the choice to select from a list of filters. Much like Memorial University's system, you will be able to list all courses in a specific subject, and from there filter out time slot conflicts, closed courses, or particular methods of delivery the user cannot participate in (a specific campus, online delivery).\par
Users should be able to provide what campus, times during which they are busy, majors they are in, etc. to the system, therefore further helping decide what courses are a good fit for the user. Once a course is added to the user's list, our system will check if the added course's time slots conflict with the user's schedule and other courses, and allow them the option to keep the course, discard the course, or to search for a better fitting section in the same course, if one exists, this of course being much better than MUNs current system that doesn't inform users of these issues until they are at the final step. \par
Additionally, users will be able to bookmark courses and view them separately. This could be useful if a user cannot decide between two conflicting courses.  With this feature, the user can bookmark both and return to them at a later date to make a decision. The users will be able to view the courses they select in both a raw listing, and a weekly schedule format, where courses would be listed in their respective time slots. \par
Another feature that we think would be useful is the ability to export the time schedule into a calendar. Users will be able to download the time table that was created based on their chosen courses. The downloaded file should be easily importable into most major calendar software, including, but not limited to, Google Calendar, Outlook, and the Apple Calendar app. \par
For users to actually sign up for courses after they have created a schedule that works for them, we simply output all CRNs they have selected, so they can head to Banner and actually sign up for the courses; although it'd be nice if we could submit this data ourselves, this is outside of the scope of this project.
Overall, we are aiming to make the experience of choosing courses more user-friendly, ergonomic, and accessible. Although it is almost certain that there will be more features, we believe that the features described so far would give us a good foundation to reach our goal.
\end{document} 

